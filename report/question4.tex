
\section{Question 4} Generative and Discriminative Subspace Learning
\newline


First, let's define the objective function. The aim of this function is to model a compromise between a generative model such as PCA and a discriminative model such as LDA.

To define this function, we first look at the objective function associated with the PCA : 
$$ J_{PCA} = \frac{1}{N} \sum_{n = 1}^{N} {\lVert x_n - \tilde{x}_n \rVert ^2} = \frac{1}{N} \sum_{n = 1}^{N}{\lVert x_n - WW^TX_n \rVert^2} $$ 
The aim is to minimise the reconstruction error.  The associated problem can therefore be written as : 
$$ min_W\ \ J_{PCA}(W) $$

We can also define the problem in the equivalent form of maximising the projected variance:
$$ J_{PCA}(W) =  \frac{1}{N} \sum_{n = 1}^{N}{\lVert W^T(x_n - \mu) \rVert^2} $$
The problem then becomes : 
$$ max_W\ \ J_{PCA}(W) $$

In the same way, we can define the objective function in the case of LDA :
$$ J_{LDA}(W) = \dfrac{W^TS_BW}{W^TS_WW} $$

Finally, the final objective function can be written as :
\begin{align*}
	J(W) &= \alpha J_{PCA} + (1 - \alpha) J_{LDA} \\
	&= \alpha \frac{1}{N} \sum_{n = 1}^{N} \lVert W^T(x_n - \mu) \rVert^2 + (1 - \alpha) \frac{W^T S_B W}{W^T S_W W}
\end{align*}

\begin{itemize}
	\item $S_B \in \mathbb{R}^{m \times m}$ is the between-class scatter
	\item $S_W \in \mathbb{R}^{m \times m}$ is the within-class scatter
	\item $\alpha$ is the balancing parameter between 0 and 1 that controls the trade-off between discrimination (0) and reconstruction (1)
	\item $x_n \in \mathbb{R}^{m}$ the nth data vector
	\item W $\in \mathbb{R}^{m \times k}$  where k is the number of eigenvectors
	\item $\mu \in \mathbb{R}^{m}$ vector of the mean of the data  $\mu = \frac{1}{N} \sum_{n = 1}^{N} x_n $ 
\end{itemize}

\subsection{Formulation of the optimisation problem}

We can rewrite the objective function in another form because we know that $ \frac{1}{N} \sum_{n = 1}^{N}{\lVert W^T(x_n - \mu) \rVert^2} = trace(W^TS_TW) $ with $S_T$ the total scatter: $$ J(W) = \alpha \ trace(W^TS_TW) + (1 - \alpha) \dfrac{W^TS_BW}{W^TS_WW}  $$

The optimisation problem is : $$max_W \ J(W)$$

\subsection{Reformulating the problem with a Lagrangian}

To facilitate optimisation, we can impose a constraint. In our case, we can impose the constraint $W^TW = 1$ since this matrix contains the unit and orthogonal eigenvectors. So, the multiplication of two vectors is either 1 if the vectors are identical (they are unitary, so their norm is 1) or 0 if they are different (since they are orthogonal).


\begin{align*}
	\mathcal{L}(W,A) &= \alpha \, \text{trace}(W^T S_T W) + (1 - \alpha) \dfrac{W^T S_B W}{W^T S_W W} \\
	& \quad - \text{trace}(A(W^T W - I))
\end{align*}


where $A$ is a matrix of Lagrange multipliers.

\subsection{Solving the problem using the gradient}

To obtain the optimum value for W, we calculate the gradient of $\mathcal{L}(W,A)$ with respect to W. Then we solve the equation in which the gradient is equal to 0.
\newline 

$\nabla_W (\alpha \ trace(W^TS_TW)) = 2\alpha S_T W$

$\nabla_W ((1 - \alpha) \dfrac{W^TS_BW}{W^TS_WW}) = (1 - \alpha) \times (\frac{2S_BW}{W^TS_WW} - \frac{2S_WW \times (W^TS_BW)}{(W^TS_WW)^2}) $

$\nabla_W (- trace(A(W^TW - I))) = -2AW $
\newline

When we put the terms together and solve the equation $\nabla_WJ(W) = 0$, we get : 
$$\alpha S_T W + (1 - \alpha) \times (\frac{S_BW}{W^TS_WW} - \frac{S_WW \times (W^TS_BW)}{(W^TS_WW)^2}) = AW$$

\subsection{Using generalized eigenvector-eigenvalues}

This gives us an equation of the form : $BW = AW$ avec $B = \alpha S_T + \times (\frac{S_B}{W^TS_WW} - \frac{S_W \times (W^TS_BW)}{(W^TS_WW)^2})$ we then need to find the generalized eigenvector of the matrix B with the constraint $W^TW = 1$

\begin{enumerate}
	\item The covariance matrices $S_B, S_W, S_T$ are calculated from the training data.
	\item B is calculated (using iterative optimisation, for example). We then look for the generalised eigenvector and corresponding generalised eigenvalues
	\item The eigenvectors associated with the largest eigenvalues are retained. These vectors will then be the columns of W.
\end{enumerate}
